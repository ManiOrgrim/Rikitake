\documentclass[a4paper, 11pt]{article}
\usepackage[T1]{fontenc}
\usepackage[utf8]{inputenc}
\usepackage{hyperref}



\begin{document}
\section*{RIKITAKE}
Rikitake is a python program that integrates the Rikitake dynamo
differential equation sysytem and calculates the Lyapunov exponents of
the system using the obtained solution.
\section*{The Rikitake dynamo}
The Rikitake dynamo is a model that describes the geodynamo, that is the
mechanism that generate the Earth's megnetic field. The first description has been made by \href{https://academic.oup.com/gji/article/35/1-3/277/615502}{Rikitake (1973)}. These equations depend on four variables $X_1$, $X_2$, $Y_1$, $Y_2$ and on two parameters $\mu$ and $k$. More informations can be found in (D. L. Turcotte, Fractals and chaos in geophysics, 1992, Cambridge University press). t is a system of four non-linear coupled differential equations that may show chaotic behaviour. A measure of the chaoticity of a system is given by the Lyapunov exponents (see, as example, \href{http://www.scholarpedia.org/article/Lyapunov_exponent}{the relative page in Scholarpedia}). 


\end{document}